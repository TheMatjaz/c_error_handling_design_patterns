% Copyright (c) 2018 Matjaž Guštin <dev@matjaz.it>
% All rights reserved.
% 
% Redistribution and use in source and binary forms, with or without
% modification, are permitted provided that the following conditions are met:
% 
% 1. Redistributions of source code must retain the above copyright notice,
%    this list of conditions and the following disclaimer.
% 2. Redistributions in binary form must reproduce the above copyright
%    notice, this list of conditions and the following disclaimer in the
%    documentation and/or other materials provided with the distribution.
% 3. Neither the name of mosquitto nor the names of its
%    contributors may be used to endorse or promote products derived from
%    this software without specific prior written permission.
% 
% THIS SOFTWARE IS PROVIDED BY THE COPYRIGHT HOLDERS AND CONTRIBUTORS "AS IS"
% AND ANY EXPRESS OR IMPLIED WARRANTIES, INCLUDING, BUT NOT LIMITED TO, THE
% IMPLIED WARRANTIES OF MERCHANTABILITY AND FITNESS FOR A PARTICULAR PURPOSE
% ARE DISCLAIMED. IN NO EVENT SHALL THE COPYRIGHT OWNER OR CONTRIBUTORS BE
% LIABLE FOR ANY DIRECT, INDIRECT, INCIDENTAL, SPECIAL, EXEMPLARY, OR
% CONSEQUENTIAL DAMAGES (INCLUDING, BUT NOT LIMITED TO, PROCUREMENT OF
% SUBSTITUTE GOODS OR SERVICES; LOSS OF USE, DATA, OR PROFITS; OR BUSINESS
% INTERRUPTION) HOWEVER CAUSED AND ON ANY THEORY OF LIABILITY, WHETHER IN
% CONTRACT, STRICT LIABILITY, OR TORT (INCLUDING NEGLIGENCE OR OTHERWISE)
% ARISING IN ANY WAY OUT OF THE USE OF THIS SOFTWARE, EVEN IF ADVISED OF THE
% POSSIBILITY OF SUCH DAMAGE.

\documentclass[aspectratio=169,14pt]{beamer}
% Copyright (c) 2018 Matjaž Guštin <dev@matjaz.it>
% All rights reserved.
% 
% Redistribution and use in source and binary forms, with or without
% modification, are permitted provided that the following conditions are met:
% 
% 1. Redistributions of source code must retain the above copyright notice,
%    this list of conditions and the following disclaimer.
% 2. Redistributions in binary form must reproduce the above copyright
%    notice, this list of conditions and the following disclaimer in the
%    documentation and/or other materials provided with the distribution.
% 3. Neither the name of mosquitto nor the names of its
%    contributors may be used to endorse or promote products derived from
%    this software without specific prior written permission.
% 
% THIS SOFTWARE IS PROVIDED BY THE COPYRIGHT HOLDERS AND CONTRIBUTORS "AS IS"
% AND ANY EXPRESS OR IMPLIED WARRANTIES, INCLUDING, BUT NOT LIMITED TO, THE
% IMPLIED WARRANTIES OF MERCHANTABILITY AND FITNESS FOR A PARTICULAR PURPOSE
% ARE DISCLAIMED. IN NO EVENT SHALL THE COPYRIGHT OWNER OR CONTRIBUTORS BE
% LIABLE FOR ANY DIRECT, INDIRECT, INCIDENTAL, SPECIAL, EXEMPLARY, OR
% CONSEQUENTIAL DAMAGES (INCLUDING, BUT NOT LIMITED TO, PROCUREMENT OF
% SUBSTITUTE GOODS OR SERVICES; LOSS OF USE, DATA, OR PROFITS; OR BUSINESS
% INTERRUPTION) HOWEVER CAUSED AND ON ANY THEORY OF LIABILITY, WHETHER IN
% CONTRACT, STRICT LIABILITY, OR TORT (INCLUDING NEGLIGENCE OR OTHERWISE)
% ARISING IN ANY WAY OUT OF THE USE OF THIS SOFTWARE, EVEN IF ADVISED OF THE
% POSSIBILITY OF SUCH DAMAGE.

\mode<presentation>

\usepackage[utf8]{inputenc} 
\usepackage[english]{babel}
\usepackage[T1]{fontenc}

\usepackage[yyyymmdd]{datetime}
\renewcommand{\dateseparator}{--}

\usepackage{xcolor} 
\definecolor{codegreen}{RGB}{0, 70, 0}
\definecolor{codegray}{RGB}{90, 90, 90}
\definecolor{codepurple}{RGB}{100, 0, 120}
\definecolor{codeblue}{RGB}{100, 0, 200}
\definecolor{codebackground}{RGB}{250, 250, 250}
\definecolor{dark-red}{rgb}{0.7,0,0}
\definecolor{dark-blue}{rgb}{0.15,0.15,0.4}
\definecolor{medium-blue}{rgb}{0,0,0.5}

\usepackage{hyperref}
\hypersetup{
    unicode=true,
    pdftoolbar=true,
    pdfmenubar=true,
    pdffitwindow=false,
    pdfstartview={FitH},
    pdftitle={Error handling design patterns in non-OOP languages},
    pdfauthor={Matjaž GUŠTIN},
    pdfsubject={Design Patterns KU assignment},
    pdfnewwindow=true,
    colorlinks=true,
    linkcolor=dark-red,
    citecolor=green,
    filecolor=magenta,
    urlcolor=medium-blue
}
 
\usepackage{listings}
\lstdefinestyle{cstyle}{
    basicstyle=\ttfamily,
    backgroundcolor=\color{codebackground},   
    commentstyle=\color{codegray},
    keywordstyle=\bf\color{blue},
    numberstyle=\tiny\color{codegray},
    stringstyle=\bf\color{codegreen},
    basicstyle=\footnotesize,
    breakatwhitespace=false,         
    breaklines=true,                 
    captionpos=b,                    
    keepspaces=true,                 
    numbers=left,                    
    numbersep=5pt,                  
    showspaces=false,                
    showstringspaces=false,
    showtabs=false,                  
    tabsize=4
}
\lstset{style=cstyle}

\newcommand{\quotes}[1]{``#1''}

\setbeamertemplate{navigation symbols}{}  % Hide navigation symbols
\setbeamercovered{invisible}

\AtBeginSection[]{
  \begin{frame}
  \vfill
  \centering
  \begin{Huge}
  \textbf{\insertsectionhead}\par%
  \end{Huge}
  \vfill
  \end{frame}
}

\title{Error handling design patterns\\in non-OOP languages}
\subtitle{Namely in ISO C}
\author{Matjaž Guštin}

\begin{document}

\begin{frame}
    \titlepage
    
    \begin{center}
        \begin{footnotesize}
        These slides are licensed under a \href{https://creativecommons.org/licenses/by/4.0/}{Creative~Commons~Attribution~4.0~International~License~(CC~BY~4.0)}.
        \end{footnotesize}
    \end{center}
\end{frame}


\begin{frame}{Overview}
    \begin{enumerate}
        \item A brief recap over Exceptions
        \item Return codes
        \item Other types of error indicators
    \end{enumerate}
    
    \begin{block}{Note}
    It will be very code-based and development-focused
    \end{block}
\end{frame}


\section{A brief recap over Exceptions}

\begin{frame}{Exceptions}
    In OOP languages we commonly have the \textbf{Exception} classes used to handle:
    \begin{itemize}
        \item unexpected values or states
        \item special cases
        \item non-nominal situations
        \item ... basically something that cannot be handled the normal way
    \end{itemize}
\end{frame}


\begin{frame}{Control flow break}
    When one Exception object is raised/thrown
    \begin{itemize}
        \item it breaks the normal execution flow
        \item makes the current function return early
        \item repeats the same on every function in the stack towards the \texttt{main()}
        \item stops when caught with a \texttt{try-catch} block
        \item or makes the program terminate abruplty if nothing catches it
    \end{itemize}
\end{frame}


\begin{frame}{Advantages of exceptions}
    \begin{itemize}
        \item The biggest advantage of exceptions is the break of the control flow
        \item It allows separation of the nominal behaviour and the error handling behaviour
        \item It's nicer to read
    \end{itemize}
\end{frame}


\begin{frame}{From \textit{Clean Code} by Robert C. Martin}
    \begin{itemize}
        \item \textit{\quotes{Returning error codes from command functions is a subtle violation of command query separation.}}
        \item \textbf{Command query separation}: \textit{\quotes{functions should either do something or answer something, but not both. Either your function should change the state of an object [command], or it should return some information about that object [query]. Doing both often leads to confusion.}}
    \end{itemize}
\end{frame}


\begin{frame}[fragile]{}
\begin{lstlisting}[style=pythonstyle]
def update_software():
    completed = False
    while not completed:
        try:
            update = download_latest_software_update()
            update.check_integrity()
            update.install()
            completed = True
        except ConnectionError:
            log.error("Retrying download later.")
            time.sleep(3600)
        except IntegrityError:
            log.error("Downloaded file corrupted.")
        except InstallationError:
            log.error("No admin rights.")
            print("Please run the program with sudo.")
            completed = True
        finally:
            update.erase_temp_files()
\end{lstlisting}
\end{frame}


\begin{frame}{But...}
    \begin{itemize}
        \item Many programming languages were created before OOP was even a thing
        \item Examples: Fortran, C, Cobol, Pascal or the more modern Rust
        \item Other patterns were found at the time and are still in use today
        \item The main difference is: \textbf{the control flow is not broken!} Manual handling is required.
        \item \textit{\quotes{If a tree falls in a forest and no one is around to hear it, does it make a sound?}}
    \end{itemize}
\end{frame}


\begin{frame}{Overview of the patterns}
    \begin{enumerate}
        \item Return codes
            \begin{enumerate}
                \item Boolean codes
                \item Error codes
                \item Status codes
                \item As output argument
            \end{enumerate}
        \item Return indicator of work done
        \item Return impossible value
        \item NULL pointers
        \item Floating point values
        \item \texttt{<errno.h>} and global error codes
        \item \texttt{<jmp.h>}
        \item SQL and NULL
    \end{enumerate}
\end{frame}


\section{Return codes}

\begin{frame}[fragile]{Boolean return code}
\begin{lstlisting}[style=cstyle]
#include <stdbool.h>

bool receive_message(message_t* message);

// Alternate version without booleans
int receive_message(message_t* message);
\end{lstlisting}

    \begin{description}
        \item[\texttt{true}] (or non-0) on success
        \item[\texttt{false}] (or 0) on failure
    \end{description}
    
    \begin{block}{The issue}
        Why did it fail? Can we retry or not? Can we tune the receving settings based on the error?
    \end{block}
\end{frame}


\section{Other error signaling systems}

\section{Conclusion}

\begin{frame}{Sources}
    \begin{itemize}
        \item \url{https://en.wikipedia.org/wiki/Exception_handling}
        \item Robert C. Martin, \textit{Clean Code: A Handbook of Agile Software Craftsmanship}, ISBN-10: 9780132350884 
    \end{itemize}
    
    \begin{footnotesize}
        \begin{center}
            These slides are licensed under a \href{https://creativecommons.org/licenses/by/4.0/}{Creative~Commons~Attribution~4.0~International~License~(CC~BY~4.0)}
            
            and the source code is available at
            
            \url{https://github.com/TheMatjaz/c_error_handling_design_patterns}
        \end{center}    
    \end{footnotesize}
\end{frame}

\end{document}
